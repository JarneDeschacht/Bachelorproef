%---------- Inleiding ---------------------------------------------------------

\section{Introductie} % The \section*{} command stops section numbering
\label{sec:introductie}
Natural Language Processing (NLP) is een term die voor velen onbekend in de oren klinkt, maar toch is het iets waar de meeste mensen dagelijks mee in contact komen. Denk maar aan een zoekmachine op het internet, virtuele assistenten en chatbots. NLP betekent de mogelijkheid om als machine menselijke taal te begrijpen en te verwerken. Tijdens deze bachelorproef zal de focus liggen op de chatbots. Chatbots zijn niet meer weg te denken uit onze maatschappij, ze hebben al meermaals aangetoond dat ze nuttig zijn bij het ondersteunen van menselijke taken en dat ze een aanzienlijke hoeveelheid werk kunnen vergemakkelijken \autocite{Atwell2007}.  De impact ervan is de voorbije jaren alleen maar gestegen en deze trend zal ook niet meteen ophouden \autocite{BRAIN2019}. Chatbots vinden vooral hun intrede bij de klantenservice van bedrijven, maar ook bij het bestellen van producten, weersvoorspellingen, advies, nieuws, enz. Er zijn tientallen tools en ontwikkelmethoden op de markt die het ontwikkelen van chatbots mogelijk maken. Bedrijven zoals Google, Facebook, Amazon, IBM en Microsoft hebben elk hun eigen NLP-platformen ontwikkeld om het creëren van chatbots voor andere bedrijven eenvoudiger te maken. Elk van deze platformen hebben hun eigen sterktes en tekortkomingen en worden continu bijgewerkt en verbeterd. Een digital product studio als In The Pocket heeft regelmatig projecten lopen met klanten waarbij het implementeren van een chatbot een toegevoerde waarde kan betekenen voor het eindresultaat. In The Pocket is officiële partner van Google Cloud en gebruikt dus voornamelijk hun producten. Het is daarom interessant voor hen om te onderzoeken hoe de verschillende cloud platformen momenteel van elkaar verschillen en waar bepaalde frameworks meer in uitblinken dan andere. Dit onderzoek kan voor hen een goed uitgangspunt zijn om in de toekomst te bepalen welk NLP-platform de beste oplossing biedt voor een bepaald project. Deze onderzoeksdoelstelling kan worden opgedeeld in een aantal specifieke deelvragen:

\begin{itemize}
  \item Met welk platform kun je de meest kwalitatieve chatbot bouwen ?
  \item Wat zijn de verschillende voordelen en tekortkomingen van bepaalde platformen en welke prijs is daar aan verbonden ?
  \item Welke tool reageert het best op user input en hoe gaat het om met taalvariaties en emoties ?
  \item Welk platform biedt de beste compatibiliteit en integraties aan ?
\end{itemize}

%---------- Stand van zaken ---------------------------------------------------

\section{State-of-the-art}
\label{sec:state-of-the-art}
Er is al veel onderzoek gedaan naar artificiële intelligentie en machineleertechnieken, maar omdat alles zo snel evolueert, is het belangrijk dat we onderzoeken blijven voeren en steeds op zoek blijven  gaan naar meer antwoorden op de problemen die zich nog stellen. Volgens  \textcite{Hussain2019} zijn er nog veel verbeterpunten mogelijk op het vlak van chatbots. Zo zou er meer focus moeten liggen op het beter verstaan van taalkundige elementen door bijvoorbeeld emotionele-en sentimentsanalyses uit te voeren en er zou ook een betere standaard moeten zijn om de kwaliteit van chatbots te testen. Een eerder onderzoek heeft aangetoond dat mensen anders communiceren als ze weten dat ze met een machine converseren. Zo zouden mensen hun taal aanpassen als ze tegen een chatbot praten, zoals mensen ook doen als ze tegen een kind bezig zijn \autocite{Hill2015}.
We kunnen concluderen dat chatbots een duidelijke meerwaarde hebben binnen onze maatschappij en dat onderzoek rond chatbots nog kan blijven verbeteren, maar hoe zit het met de onderlinge vergelijking tussen verschillende chatbots ? Zijn bepaalde chatbots beter dan andere ? Volgens het onderzoek van \textcite{Russis2018} is de tool die IBM (Watson) aanbiedt om chatbots in de cloud te bouwen de beste op de markt met als dichte achtervolgers Microsoft (LUIS) en Google (Dialogflow). Dit wordt betwist door het onderzoek van \textcite{Langen2017}, want zij besluiten dat LUIS (Microsoft) veruit het beste platform is. Deze platformen evolueren continu, net zoals de noden en eisen van klanten, dus een vergelijkende studie anno 2020 is dan ook zeker interessant. 


% Voor literatuurverwijzingen zijn er twee belangrijke commando's:
% \autocite{KEY} => (Auteur, jaartal) Gebruik dit als de naam van de auteur
%   geen onderdeel is van de zin.
% \textcite{KEY} => Auteur (jaartal)  Gebruik dit als de auteursnaam wel een
%   functie heeft in de zin (bv. ``Uit onderzoek door Doll & Hill (1954) bleek
%   ...'')
%---------- Methodologie ------------------------------------------------------
\section{Methodologie}
\label{sec:methodologie}

Bij het uitwerken van deze bachelorproef kan het volledige proces worden opgedeeld in 5 fases.
In de eerste fase zal er algemene uitleg rond chatbots plaatsvinden en zullen belangrijke begrippen toegelicht worden. Er zal ook stilgestaan worden bij de uitdagingen en tekortkomingen van chatbots op dit moment. Tijdens de tweede fase zal er een vergelijkende studie worden uitgevoerd, waarbij er een aantal NLP-platformen zullen worden vergeleken met elkaar. Ze zullen worden vergeleken op vaste criteria zoals prijs, integratiemogelijkheden, complexiteit, ontwikkeltijd, verschillende (programmeer) talen, ... De belangrijkste voor-en nadelen zullen ook uitvoerig besproken worden. Er zal hier niet alleen naar de grote marktspelers worden gekeken, maar ook naar kleinere platformen die minder in de belangstelling staan zoals Recast.ai, ChatFuel, RASA, AgentBot, …
Tijdens de derde fase zullen we de drie beste platformen selecteren en zal er een test scenario worden opgesteld, waarbij er een aantal factoren van deze tools uitvoerig zullen getest en geëvalueerd zullen worden. Deze case zal de volgende aspecten van de gebouwde chatbots zeer uitvoerig gaan testen en evalueren:

\begin{itemize}
    \item Hoe reageert de chatbot op een normale eenvoudige conversatie met weinig complexe expressies ?
    \item Hoe gaat de chatbot om met negatieve expressies ? Dit kunnen klantenklachten zijn of negatieve emoties
    \item Hoe goed neemt de chatbot slecht gevormde zinnen en woorden op en hoe accuraat is het antwoord ?
    \item Hoe accuraat blijft het antwoord van de chatbot als hij erg complexe zinnen moet verwerken ?
    \item Hoe goed gaat de chatbot om met berichten die niets hebben te maken met de scenario’s waarvoor hij is getraind ?
\end{itemize}

\bigskip
Tijdens de volgende fase van het proces zullen de chatbots die gebouwd zijn met de top drie platformen het testscenario doorlopen en zullen alle resultaten geanalyseerd worden.
De vijfde en laatste fase zal dienen voor de uiteindelijke vergelijking van alle resultaten die werden bekomen met de vorige fasen en zal er een conclusie gevormd worden.


%---------- Verwachte resultaten ----------------------------------------------
\section{Verwachte resultaten}
\label{sec:verwachte_resultaten} 

Een verwachting is dat er geen concrete voorkeur zal zijn voor een platform in verband met simpele conversaties met korte en duidelijke zinnen. Als het aankomt op nieuwe trainingsdata, dan is de verwachting dat LUIS het beste zal presteren door de active learning technologie die er in geïmplementeerd is. DialogFlow zou volgens eerder onderzoek dan weer het beste presteren op het herkennen van berichten die niets te maken hebben met waarvoor de chatbot is getraind. Op accuraatheid zou volgens eerdere vaststellingen LUIS het beste functioneren \autocite{Russis2018}. Daarentegen wordt er verwacht dat de integratiemogelijkheden van LUIS, Watson en DialogFlow hoger scoren dan van de andere concurrenten, omdat dit de grotere marktspelers zijn op vlak van chatbotplatformen.

%---------- Verwachte conclusies ----------------------------------------------
\section{Verwachte conclusies}
\label{sec:verwachte_conclusies}

Uit dit onderzoek verwachten we te kunnen concluderen dat het wel degelijk interessant is voor In The Pocket om in bepaalde scenario’s over te stappen naar een alternatief in plaats van Google producten. Indien er een beperkt budget is, dan verwachten we dat het nog steeds mogelijk is om door te werken met Dialogflow of Wit.ai omdat deze gratis zijn. Op het moment dat de complexiteit toeneemt en de accuraatheid prioriteit wordt, dan zou men beter overstappen naar LUIS (Microsoft). Bij de keuze naar een platform met veel integratiemogelijkheden zou de voorkeur uitgaan naar DialogFlow, LUIS of Watson, maar opnieuw zal het budget daar een belangrijke factor in zijn.

