%---------- Inleiding ---------------------------------------------------------

\section{Introductie} % The \section*{} command stops section numbering
\label{sec:introductie}
Natural Language Processing (NLP) is een term die voor velen onbekend in de oren klinkt, maar toch is het iets waar de meeste mensen dagelijks mee in contact komen. Denk maar aan een zoekmachine op het internet, virtuele assistenten en chatbots. NLP betekent de mogelijkheid om als machine menselijke taal te begrijpen en te verwerken. Tijdens deze bachelorproef zal de focus liggen op de chatbots. Chatbots zijn niet meer weg te denken uit onze maatschappij, ze hebben al meermaals aangetoond dat ze nuttig zijn bij het ondersteunen van menselijke taken en dat ze een aanzienlijke hoeveelheid werk kunnen vergemakkelijken \autocite{Atwell2007}.  De impact ervan is de voorbije jaren alleen maar gestegen en deze trend zal ook niet meteen ophouden \autocite{BRAIN2019}. Chatbots vinden vooral hun intrede bij de klantenservice van bedrijven, maar ook bij het bestellen van producten, weersvoorspellingen, advies, nieuws, enz. Er zijn tientallen tools en ontwikkelmethoden op de markt die het ontwikkelen van chatbots mogelijk maken. Bedrijven zoals Google, Facebook, Amazon, IBM en Microsoft hebben elk hun eigen NLP-platform ontwikkeld om het creëren van chatbots voor andere bedrijven eenvoudiger te maken. Elk van deze platformen hebben hun eigen sterktes en tekortkomingen en worden continu bijgewerkt en verbeterd. Een digital product studio als In The Pocket heeft regelmatig projecten lopen met klanten waarbij het implementeren van een chatbot een toegevoerde waarde kan betekenen voor het eindresultaat. In The Pocket is officiële partner van Google Cloud en gebruikt dus voornamelijk hun producten. Het is daarom interessant voor hen om te onderzoeken hoe de verschillende cloud platformen momenteel van elkaar verschillen en waar bepaalde frameworks meer in uitblinken dan andere. Dit onderzoek kan voor hen een goed uitgangspunt zijn om in de toekomst te bepalen welk NLP-platform de beste oplossing biedt voor een bepaald project. Deze onderzoeksdoelstelling kan worden opgedeeld in een aantal specifieke deelvragen:
\bigskip
\begin{itemize}
  \item Welk chatbotplatform heeft het beste begrijpend vermogen om berichten correct te interpreteren en accuraat te antwoorden ? 
  \begin{itemize}
      \item Welk platform is het meest interessant indien er een beperkt budget is ?
  \end{itemize}
  \item Wat zijn de voordelen en tekortkomingen van de verschillende platformen ?
  \item Welk platform biedt de meeste integraties met andere diensten aan ?
\end{itemize}

%---------- Stand van zaken ---------------------------------------------------

\section{State-of-the-art}
\label{sec:state-of-the-art}
Er is al veel onderzoek gedaan naar artificiële intelligentie en machineleertechnieken, maar omdat alles zo snel evolueert, is het belangrijk dat we onderzoeken blijven voeren en steeds op zoek blijven  gaan naar meer antwoorden op de problemen die zich nog stellen. Volgens  \textcite{Hussain2019} zijn er nog veel verbeterpunten mogelijk op het vlak van chatbots. Zo zou er meer focus moeten liggen op het beter verstaan van taalkundige elementen door bijvoorbeeld emotionele-en sentimentsanalyses uit te voeren en er zou ook een betere standaard moeten zijn om de kwaliteit van chatbots te testen. Een eerder onderzoek heeft aangetoond dat mensen anders communiceren als ze weten dat ze met een machine converseren. Zo zouden mensen hun taal aanpassen als ze tegen een chatbot praten, zoals mensen ook doen als ze tegen een kind bezig zijn \autocite{Hill2015}.
We kunnen concluderen dat chatbots een duidelijke meerwaarde hebben binnen onze maatschappij en dat onderzoek rond chatbots nog kan blijven verbeteren, maar hoe zit het met de onderlinge vergelijking tussen verschillende chatbots ? Zijn bepaalde chatbots beter dan andere ? Volgens het onderzoek van \textcite{Russis2018} is de tool die IBM (Watson) aanbiedt om chatbots in de cloud te bouwen de beste op de markt met als dichte achtervolgers Microsoft (LUIS) en Google (Dialogflow). Dit wordt betwist door het onderzoek van \textcite{Langen2017}, want zij besluiten dat LUIS (Microsoft) veruit het beste platform is. In 2019 is er echter veel veranderd in de wereld van NLP. Dat komt door de modellen die gebaseerd zijn op de transformers. Een voorbeeld van zo’n model is BERT, die ontwikkeld is door Google. Het concept van een transformer zorgt er voor dat relaties tussen woorden in zinnen beter kunnen gevonden worden door machines \autocite{Joshi2019}. Chatbotplatformen gebruiken deze modellen om betere resultaten op te leveren. Een voorbeeld hiervan is Reply.ai, die BERT sterk gebruikt. Doordat transformers zo’n grote intrede gemaakt hebben, zijn vele voorgaande studies over welk platform de voorkeur krijgt geneutraliseerd. Een vergelijkende studie anno 2020 is dan ook zeker interessant om terug een beter overzicht van de actualiteit te krijgen.


% Voor literatuurverwijzingen zijn er twee belangrijke commando's:
% \autocite{KEY} => (Auteur, jaartal) Gebruik dit als de naam van de auteur
%   geen onderdeel is van de zin.
% \textcite{KEY} => Auteur (jaartal)  Gebruik dit als de auteursnaam wel een
%   functie heeft in de zin (bv. ``Uit onderzoek door Doll & Hill (1954) bleek
%   ...'')
%---------- Methodologie ------------------------------------------------------
\section{Methodologie}
\label{sec:methodologie}

Bij het uitwerken van deze bachelorproef kan het volledige proces worden opgedeeld in 5 fases.
In de eerste fase zal er algemene uitleg rond chatbots plaatsvinden en zullen belangrijke begrippen toegelicht worden. Er zal ook stilgestaan worden bij de uitdagingen en tekortkomingen van chatbots op dit moment. Tijdens de tweede fase zal er een vergelijkende studie worden uitgevoerd, waarbij er een aantal NLP-platformen zullen worden vergeleken met elkaar. Ze zullen worden vergeleken op vaste criteria zoals prijs, integratiemogelijkheden, complexiteit, ontwikkeltijd, ondersteuning voor verschillende (programmeer) talen, pre-built entities/intents, spraakherkenning, ... De belangrijkste voor-en nadelen zullen ook uitvoerig besproken worden. Er zal hier niet alleen naar de grote marktspelers worden gekeken, maar ook naar kleinere platformen die minder in de belangstelling staan zoals Recast.ai, ChatFuel, RASA, AgentBot, Botsify, Reply.ai, … Tijdens de derde fase zullen de drie meest interessante platformen uit deze lijst geselecteerd worden en zal er een testscenario worden opgesteld waarbij er een aantal factoren van deze 3 gekozen tools uitvoerig getest en geëvalueerd zullen worden. Voor het testscenario zal er een dataset voorzien worden met testdata waarmee elk gekozen platform zal getraind worden. Daarna zullen de chatbots beoordeeld worden op de verschillende criteria door middel van conversaties waarbij er zal worden gekeken naar de volgende criteria:\bigskip

\begin{itemize}
    \item Hoe reageert de chatbot op een normale eenvoudige conversatie met weinig moeilijke zinnen ?
    \item Hoe gaat de chatbot om met negatieve expressies ? Dit kunnen klantenklachten zijn of negatieve emoties.
    \item Hoe goed neemt de chatbot slecht gevormde zinnen en woorden op en hoe accuraat is het antwoord ?
    \item Hoe accuraat blijft het antwoord van de chatbot als hij erg complexe zinstructuren moet verwerken ? Dit kunnen zinnen zijn die heel erg lang zijn, veel moeilijke woorden en werkwoorden bevatten, vragen en bevelen op het zelfde moment bevatten, quote's bevatten, ...
    \item Hoe goed gaat de chatbot om met berichten die niets hebben te maken met de scenario’s waarvoor hij is getraind ?
    \item Hoe goed herkent de chatbot het verschil tussen een vraag en een bevel ?
\end{itemize}

\bigskip
Tijdens de volgende fase van het proces zullen de chatbots die gebouwd zijn met de top drie platformen het testscenario doorlopen en zullen alle resultaten geanalyseerd worden.
De vijfde en laatste fase zal dienen voor de uiteindelijke vergelijking van alle resultaten die werden bekomen met de vorige fasen en zal er een conclusie gevormd worden.


%---------- Verwachte resultaten ----------------------------------------------
\section{Verwachte resultaten}
\label{sec:verwachte_resultaten} 

Een verwachting is dat er geen concrete voorkeur zal zijn voor een platform in verband met simpele conversaties met korte en duidelijke zinnen. Als het aankomt op nieuwe trainingsdata, dan is de verwachting dat LUIS het beste zal presteren door de active learning technologie die er in geïmplementeerd is. DialogFlow zou volgens eerder onderzoek dan weer het beste presteren op het herkennen van berichten die niets te maken hebben met waarvoor de chatbot is getraind en op accuraatheid zou LUIS de beste resultaten moeten kunnen voorleggen \autocite{Russis2018}. Op accuraatheid zou ook Reply.ai goed moeten scoren door het sterk gebruik van BERT. Daarentegen wordt er verwacht dat de integratiemogelijkheden van LUIS, Watson en DialogFlow hoger scoren dan van de andere concurrenten, omdat dit de grotere marktspelers zijn op vlak van chatbotplatformen.

%---------- Verwachte conclusies ----------------------------------------------
\section{Verwachte conclusies}
\label{sec:verwachte_conclusies}

Uit dit onderzoek verwachten we te kunnen concluderen dat het wel degelijk interessant is voor In The Pocket om in bepaalde scenario’s over te stappen naar een alternatief in plaats van Google producten. Indien er een beperkt budget is, dan verwachten we dat Dialogflow en Wit.ai de interessantste keuzes zullen zijn, omdat deze gratis zijn. Op het moment dat het begrijpend vermogen om berichten correct te interpreteren en accuraat te antwoorden prioriteit wordt, dan wordt er verwacht dat het beter is om over te stappen naar LUIS (Microsoft) of Reply.ai. Bij de keuze naar een platform met veel integratiemogelijkheden zou de voorkeur uitgaan naar DialogFlow, LUIS of Watson.

