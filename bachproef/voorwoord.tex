%%=============================================================================
%% Voorwoord
%%=============================================================================

\chapter*{\IfLanguageName{dutch}{Woord vooraf}{Preface}}
\label{ch:voorwoord}

%% TODO:
%% Het voorwoord is het enige deel van de bachelorproef waar je vanuit je
%% eigen standpunt (``ik-vorm'') mag schrijven. Je kan hier bv. motiveren
%% waarom jij het onderwerp wil bespreken.
%% Vergeet ook niet te bedanken wie je geholpen/gesteund/... heeft

Ik wil tijdens mijn professionele carrière met zoveel mogelijk technologieën in aanraking komen, omdat ik daar nieuwsgierig in ben, maar ook om zeker te zijn waar mijn grootste interesse ligt. Tijdens mijn opleiding toegepaste informatica aan Hogeschool Gent heb ik nauwelijks inzicht gekregen in de wereld van artificiële intelligentie en machine learning en dat vond ik enorm jammer. Er was een opleidingsonderdeel in het derde jaar rond artificiële intelligentie, maar dat heb ik niet gekregen doordat ik dat semester heb gekozen voor een periode in het buitenland. Tijdens deze periode heb ik dat vak op eigen initiatief wat bekeken en daar ontstond het idee om mijn bachelorproef te doen rond artificiële intelligentie, of toch een subdomein daarvan. De keuze om iets rond chatbots te doen kwam vrij snel, omdat ik altijd wel al geïnteresseerd was in hoe dat juist functioneerde en hoe het toch kon dat een machine menselijke taal kan verstaan en zelf kan genereren. Het leek me heel tof en leerrijk om dat zelf te kunnen bouwen. De bachelorproef leek me een ideaal moment om daar onderzoek rond te doen en mijn kennis daarin te verbeteren.

Ik zou graag mijn promotor Liesbeth Lewyllie willen bedanken voor haar begeleiding en steun. Ze heeft me inhoudelijk en tijdens de uitwerking van dit onderzoek continu begeleid en telkens de nodige feedback voorzien. Bij vragen en onduidelijkheden rond eender wat kon ik bij haar terecht en heeft ze me steeds goed verder geholpen.

Daarnaast wil ik ook mijn co-promotor Kenny Helsens bedanken om me vanaf het begin te ondersteunen in mijn idee. We hebben samen de scope van dit onderzoek vastgelegd en hij heeft me de vrijheid gegeven om een bachelorproef te maken waarin ik zelf veel beslissingen moest maken. Zijn persoonlijke kennis en ervaring met het bedrijf In The Pocket hebben een duidelijke bijdrage in het eindresultaat.

Als laatste wil ik mijn vriendin Ime Van Daele bedanken om mijn bachelorproef meerdere malen te lezen en te verbeteren op grammaticale fouten. Alhoewel ze niet thuis is binnen dit onderzoekdomein, heeft ze dit werk met de nodige kritieke blik bekeken en veel nuttige feedback gegeven, wat zeker een positieve impact heeft.
