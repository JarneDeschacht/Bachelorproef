%%=============================================================================
%% Samenvatting
%%=============================================================================

%  De "abstract" of samenvatting is een kernachtige (~ 1 blz. voor een
% thesis) synthese van het document.
%
% Deze aspecten moeten zeker aan bod komen:
% - Context: waarom is dit werk belangrijk?
% - Nood: waarom moest dit onderzocht worden?
% - Taak: wat heb je precies gedaan?
% - Object: wat staat in dit document geschreven?
% - Resultaat: wat was het resultaat?
% - Conclusie: wat is/zijn de belangrijkste conclusie(s)?
% - Perspectief: blijven er nog vragen open die in de toekomst nog kunnen
%    onderzocht worden? Wat is een mogelijk vervolg voor jouw onderzoek?
%
% LET OP! Een samenvatting is GEEN voorwoord!

%%---------- Nederlandse samenvatting -----------------------------------------
%
%  Als je je bachelorproef in het Engels schrijft, moet je eerst een
% Nederlandse samenvatting invoegen. Haal daarvoor onderstaande code uit
% commentaar.
% Wie zijn bachelorproef in het Nederlands schrijft, kan dit negeren, de inhoud
% wordt niet in het document ingevoegd.

\IfLanguageName{english}{%
\selectlanguage{dutch}
\chapter*{Samenvatting}
\lipsum[1-4]
\selectlanguage{english}
}{}

%%---------- Samenvatting -----------------------------------------------------
% De samenvatting in de hoofdtaal van het document

\chapter*{\IfLanguageName{dutch}{Samenvatting}{Abstract}}

Chatbots zijn tegenwoordig niet meer weg te denken uit onze maatschappij. We beseffen het niet altijd, maar we hebben meer interactie met een chatbot dan dat we initieel zouden verwachten. Denk hierbij bijvoorbeeld aan de klantenservice van bedrijven, want het gebeurt niet vaak meer dat ons eerste contact direct met een menselijke medewerker is als we bijvoorbeeld een chatbericht sturen naar de klantenservice van een bedrijf. Meestal is er een geautomatiseerde gesprekspartner geïmplementeerd om de klant verder te helpen bij problemen en vragen. Bedrijven gaan tegenwoordig vaak hun eigen chatbots gaan bouwen, maar hoe moet dat juist ? De ontwikkelingen binnen de wereld van artificiële intelligentie zijn de laatste jaren enorm sterk toegenomen. Deze ontwikkelingen bieden bedrijven de mogelijkheid om daarop in te spelen en er zelf mee te werken. Het implementeren van een chatbot vanaf het begin is een erg complex proces die veel tijd en kennis vereisen. Verschillende bedrijven hebben hierop ingespeeld door platformen te ontwikkelen die de implementatie van een chatbot vergemakkelijken. Hierdoor kunnen bedrijven, waar er onvoldoende kennis en tijd is voor de implementatie van chatbots, toch een chatbot bouwen en gebruiken. Er zijn tal van bedrijven die hun eigen chatbotplatformen hebben ontwikkeld waarmee eenvoudig een chatbot opgezet kan worden. Bedrijven zoals Google, Facebook, Microsoft, IBM, Amazon, etc. hebben allemaal zo’n platform gebouwd, maar welk van deze platformen is de beste ?

Binnen deze bachelorproef is een uitgebreid onderzoek gevoerd naar de verschillende gratis platformen die overweg kunnen met de Nederlandse taal, de meest interessante werden verder uitgediept. Daarbij werd er stilgestaan bij de verschillende functionaliteiten, de voor- en nadelen, de werking, de presentaties en de verschillende integratiemogelijkheden, zoals bijvoorbeeld met Slack of Facebook Messenger. De focus binnen deze bachelorproef ligt op het vinden van het platform die het best verstaat wat de gebruiker juist bedoeld en ook belangrijke informatie zoals datums, locaties, tijdstippen, etc. kan extraheren. Daarbij wordt ook rekening gehouden met spellingsfouten. De platformen zijn door middel van objectieve beoordelingstechnieken en verschillende experimenten getest. 

Het resultaat is dat het platform van Facebook (Wit.ai) algemeen de beste resultaten kan voorleggen. Zowel bij het correct verstaan van berichten als bij het afleiden van belangrijke informatie, scoort deze het sterkst. Het nadeel van Wit.ai is dat het moeite heeft met spellingsfouten en scoort daarbij minder goed in vergelijking met zijn concurrenten. Ook de ingebouwde integratiemogelijkheden van andere platformen zijn geavanceerder.

Deze bachelorproef biedt veel mogelijkheden tot verder onderzoek doordat de beoordeling werd beperkt tot de Nederlandse taal en doordat er geen betaalde platformen onderzocht werden. De dataset die werd gebruikt, is ook fictief en te klein om perfect representatief te zijn. Een onderzoek met een grotere representatieve dataset zou hier een oplossing voor kunnen bieden. Dit onderzoek is alvast een goed startpunt.





























