%%=============================================================================
%% Samenvatting
%%=============================================================================

%  De "abstract" of samenvatting is een kernachtige (~ 1 blz. voor een
% thesis) synthese van het document.
%
% Deze aspecten moeten zeker aan bod komen:
% - Context: waarom is dit werk belangrijk?
% - Nood: waarom moest dit onderzocht worden?
% - Taak: wat heb je precies gedaan?
% - Object: wat staat in dit document geschreven?
% - Resultaat: wat was het resultaat?
% - Conclusie: wat is/zijn de belangrijkste conclusie(s)?
% - Perspectief: blijven er nog vragen open die in de toekomst nog kunnen
%    onderzocht worden? Wat is een mogelijk vervolg voor jouw onderzoek?
%
% LET OP! Een samenvatting is GEEN voorwoord!

%%---------- Nederlandse samenvatting -----------------------------------------
%
%  Als je je bachelorproef in het Engels schrijft, moet je eerst een
% Nederlandse samenvatting invoegen. Haal daarvoor onderstaande code uit
% commentaar.
% Wie zijn bachelorproef in het Nederlands schrijft, kan dit negeren, de inhoud
% wordt niet in het document ingevoegd.

\IfLanguageName{english}{%
\selectlanguage{dutch}
\chapter*{Samenvatting}
\lipsum[1-4]
\selectlanguage{english}
}{}

%%---------- Samenvatting -----------------------------------------------------
% De samenvatting in de hoofdtaal van het document

\chapter*{\IfLanguageName{dutch}{Samenvatting}{Abstract}}

Chatbots zijn tegenwoordig niet meer weg te denken uit onze maatschappij. We beseffen het misschien niet altijd, maar we hebben meer interactie met een chatbot dan dat we initieel zouden verwachten. Denk maar aan de klantenservice van bedrijven bijvoorbeeld. Het gebeurt niet zo vaak meer dat ons eerste contact direct met een menselijke medewerker is. Meestal zit er een geautomatiseerde gesprekspartner te wachten om de klant verder te helpen bij problemen en vragen. Bedrijven gaan tegenwoordig vaak hun eigen chatbots gaan bouwen, maar hoe moet dat juist ? De ontwikkelingen binnen de wereld van artificiële intelligentie zijn de laatste jaren enorm sterk toegenomen en bieden daarom bedrijven de mogelijkheid om daar op in te spelen en daar zelf mee te gaan werken. Het implementeren van een chatbot vanaf nul is een heel erg complexe zaak en daar hebben verschillende bedrijven op ingespeeld door platformen te bouwen die dat proces vergemakkelijken. Bedrijven waarin de kennis om dat vanaf nul op te zetten niet aanwezig is of waarin de focus op andere zaken ligt kunnen op deze manier toch chatbots gaan bouwen en gebruiken. Er zijn tal van bedrijven die hun eigen chatbotplatformen hebben ontwikkeld waarmee eenvoudig een chatbot ontwikkeld kan worden. Bedrijven zoals Google, Facebook, Microsoft, IBM, Amazon, etc. hebben allemaal zo’n platform gebouwd, maar welk van deze platformen is de beste oplossing ?

Binnen deze bachelorproef is een uitgebreid onderzoek gevoerd naar de verschillende gratis platformen die perfect overweg kunnen met de Nederlandse taal en werden de meest interessante verder uitgediept. Daarbij wordt er stilgestaan bij de verschillende functionaliteiten, voor-en nadelen, integratiemogelijkheden met bijvoorbeeld Slack of Facebook Messenger, werking en prestaties. De focus binnen deze bachelorproef ligt op het vinden van het platform die het beste kan verstaan wat de gebruiker juist bedoeld en ook belangrijke informatie zoals datums, locaties, tijdstippen, etc. kan extraheren. Daarbij wordt ook rekening gehouden met spellingsfouten. De platformen zijn door middel van objectieve beoordelingstechnieken en verschillende experimenten uitbundig getest. 

Het resultaat is dat het platform van Facebook (Wit.ai) algemeen de beste resultaten kan voorleggen. Zowel bij het correct verstaan van berichten als het afleiden van belangrijke informatie scoort Wit.ai het sterkst. Het platform heeft wel moeite met spellingsfouten en scoort daarbij niet goed in vergelijking met zijn concurrenten. Ook de ingebouwde integratiemogelijkheden van andere platformen zijn geavanceerder.

Deze bachelorproef biedt heel wat mogelijkheden tot verder onderzoek doordat er beperkt werd tot de Nederlandse taal en dat er geen betaalde platformen onderzocht werden. De dataset die werd gebruikt is ook fictief en te klein om volledig representatief te zijn. Daar zou verder onderzoek met een grote representatieve dataset een oplossing voor bieden. Dit onderzoek biedt alvast een goed startpunt.





























