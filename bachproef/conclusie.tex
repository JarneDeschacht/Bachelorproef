%%=============================================================================
%% Conclusie
%%=============================================================================

\chapter{Conclusie}
\label{ch:conclusie}

% Trek een duidelijke conclusie, in de vorm van een antwoord op de
% onderzoeksvra(a)g(en). Wat was jouw bijdrage aan het onderzoeksdomein en
% hoe biedt dit meerwaarde aan het vakgebied/doelgroep? 
% Reflecteer kritisch over het resultaat. In Engelse teksten wordt deze sectie
% ``Discussion'' genoemd. Had je deze uitkomst verwacht? Zijn er zaken die nog
% niet duidelijk zijn?
% Heeft het onderzoek geleid tot nieuwe vragen die uitnodigen tot verder 
%onderzoek?

Uit dit onderzoek kan geconcludeerd worden dat de resultaten van de verschillende platformen voor intentherkenning heel erg dicht bij elkaar liggen. Dit zorgt ervoor dat het moeilijk te besluiten is welk platform het beste presteert, aangezien dit op enkele procenten aankomt. Wit.ai kan de beste resultaten voorleggen met een verschil van minder dan 1 procent op IBM Watson. Wit.ai scoort wel slechter als er veel spellingsfouten in de voorbeeldzinnen staan. Daar zien we dat IBM Watson de duidelijke winnaar is en dat Wit.ai bijna onderaan staat.

Op het vlak van entityherkenning is het wel duidelijk, Wit.ai steekt er bovenuit en niemand komt dicht in de buurt. Dialogflow presteert ook goed, maar niet goed genoeg om waardige concurrentie te zijn voor Wit.ai. Bij zowel het gebruik van spellingsfouten als zonder, is Wit.ai duidelijk de beste.

Wit.ai is vrij gemakkelijk te gebruiken en heeft als grote voordelen dat het volledig gratis te gebruiken is en dat het bijna alle talen ondersteund. De documentatie voor zowel de grafische interface als de API is ook goed uitgewerkt. De API werkt wel niet optimaal. Het is een aantal keer voorgevallen dat niet alle data die werd verstuurd, ook effectief in het platform terecht kwam, terwijl de API wel een bericht terug stuurde dat alles succesvol was toegevoegd. Het duurde vaak lang tot de applicatie getraind was, dit kan problemen veroorzaken bij applicaties die groter zijn en meer data bevatten.

Dialogflow heeft de beste integratiemogelijkheden. Bij Wit.ai zijn deze beperkt, doordat het deel uitmaakt van Facebook. Het is natuurlijk wel altijd mogelijk om zelf een applicatie op te zetten door gebruik te maken van de API’s en SDK’s.

Alhoewel dat er objectieve beoordelingstechnieken gebruikt zijn, geeft dit onderzoek enkel een indicatie van welk platform het beste presteert. Er is een fictieve dataset opgesteld waarin een aantal voorbeeldzinnen werden toegevoegd die gebruikt werden. Deze dataset is te klein om een sluitend antwoord te formuleren over welk platform de beste is. Een representatieve dataset opstellen was in dit onderzoek niet mogelijk doordat er een beperkte tijd is. Er is ook beperkt tot de (grotendeels) gratis platformen, wat dus niet kan uitsluiten dat er bepaalde betalende platformen beter presteren. Dit onderzoek is ook volledig op Nederlandse data uitgevoerd, maar sommige platformen staan daarbij nog niet zo ver als waar ze staan bij de Engelse taal.

Dit onderzoek biedt een mooi startpunt voor verder onderzoek, waarin er gebruik gemaakt kan worden van representatieve datasets om te beoordelen of er daar andere conclusies worden bekomen. Het is ook interessant om te onderzoeken of dat er grote verschillen zijn in resultaten als er Engelse data gebruikt wordt. Daarnaast is het ook een mooi startpunt om dit onderzoek verder uit te breiden met betalende platformen om de verschillen vast te leggen.


